\documentclass[report.tex]{subfiles}

\begin{document}

\chapter{The toy example Poisson process time simulation}
\label{appendix-toy-example-poisson-process-jump-time-simulation}

We will derive exact formulas for simulating the Poisson process times
for the toy PDMP from Section~\ref{pdmp-toy-example-section}.
Let the state space be
$Z = \mathbb{R}^{2d} = \mathbb{R}^{d} \times \mathbb{R}^{d} = (X, Y)$.
We have the flow function given by $\phi((x,v), t) = (x + vt, v)$ and the
intensity function given by $\lambda(x, v) = \max\{0, \left\langle x, v \right\rangle\}$
We give all the credit to \citet{bouchard2015bouncy} for the derivation
of the following results.

Note that we can write (with $\lVert \cdot \rVert$ denoting the $L^{2}$ norm):

\begin{align*}
  \lambda(\phi((x, v), t))
  &= \lambda(x + vt, v) \\
  &= \langle x + vt, v \rangle^{+} \\
  &= \left\langle \nabla \frac{\lVert x + vt \rVert^{2}}{2}, \dv{}{t} (x + vt) \right\rangle^{+} \\
  &= \left( \dv{}{t} \frac{\lVert x + vt \rVert^{2}}{2} \right)^{+}.
\end{align*}

Now note that $\lVert \cdot \rVert$ is a strictly convex function. It thus follows, that
for a given $(x, v) \in \mathbb{R}^{2d}\,\, \exists \tau^{*} \geq 0$, such that
$\dv{}{t} \lVert x + vt \rVert^{2}$ is negative for $t \in (0, \tau^{*})$ and positive
for $t \in (\tau^{*}, \infty)$. We can calculate the value of $\tau^{*}$ exactly,
by solving $\dv{}{t} \lVert x + vt \rVert^{2} = 0$ for $t$ (and setting $\tau^{*} = 0$ if the solution is negative).
One can easily verify that:
\begin{equation}
  \label{tau-star-equation}
  \tau^{*} = \left( -\frac{\langle x, v \rangle}{\lVert v \rVert^{2}} \right)^{+}.
\end{equation}

Now lets see how we can use the Algorithm~\ref{alg-non-hom-pp-time-scale}
(the time scale tr  ansformation algorithm)
for simulating jump times of our PDMP. We first simulate
$U \sim \text{Uniform}(0, 1)$
and try to find smallest $\tau \geq 0$ such that
$$
\myint{0}{\tau}{\lambda(\phi((x, v), t))}{t}
= \myint{0}{\tau}{\left( \dv{}{t} \frac{\lVert x + vt \rVert^{2}}{2} \right)^{+}}{t}
= -\log U.
$$
Now note that we can rewrite:
\begin{align*}
  \myint{0}{\tau}{\left( \dv{}{t} \frac{\lVert x + vt \rVert^{2}}{2} \right)^{+}}{t}
  &= \myint{0}{\tau^{*}}{0}{t}
  + \myint{\tau^{*}}{\tau}{\dv{}{t} \frac{\lVert x + vt \rVert^{2}}{2}}{t} \\
  &= \myint{\tau^{*}}{\tau}{\dv{}{t} \frac{\lVert x + vt \rVert^{2}}{2}}{t} \\
  &= \frac{\lVert x + v\tau \rVert^{2}}{2} - \frac{\lVert x + v\tau^{*} \rVert^{2}}{2} \\
\end{align*}
Hence, to get the jump time we need to solve the following equation for $\tau$:
$$
  \frac{\lVert x + v\tau \rVert^{2}}{2} - \frac{\lVert x + v\tau^{*} \rVert^{2}}{2}
  = -\log U
$$
If $\tau^{*} = 0$, the equation becomes:
$$
\lVert v \rVert^{2} \tau^{2} + 2 \langle x, v \rangle \tau + 2 \log U = 0
$$
and solving for the positive root, we obtain:
$$
\tau = \frac{1}{\lVert v \rVert^{2}} \left( - \langle x, v \rangle + \sqrt{\langle x, v \rangle^{2} - 2 \lVert v \rVert^{2} \log U} \right).
$$
If $\tau^{*} > 0$, ten by using Equation~\ref{tau-star-equation} we get the following quadratic equation:
$$
\lVert v \rVert^{2} \tau^{2} + 2 \langle x, v \rangle \tau + 2 \log U - \frac{\langle x, v \rangle^{2}}{\lVert v  \rVert^{2}}= 0
$$
with the positive root giving:
$$
\tau = \frac{1}{\lVert v \rVert^{2}} \left( - \langle x, v \rangle + \sqrt{- 2 \lVert v \rVert^{2} \log U} \right).
$$
This concludes the derivation of equations for the jump time simulation of the toy PDMP from Section~\ref{pdmp-toy-example-section}.



\end{document}
