\documentclass[report.tex]{subfiles}

\begin{document}

\chapter{Conclusion}

In Section~\ref{project-goals-section} we have set the goal to build an
\textit{efficient}, \textit{flexible},
\textit{reliable}, \textit{relevant} and \textit{easy to use}
framework for PDMPs simulation which will hopefully speed up future research
on MCMC methods. We will now briefly review how each of these 5 qualities were
achieved:
\begin{enumerate}
  \item We addressed \textit{efficiency} of our framework in Chapter~\ref{pdmps-implementation}
    by choosing the right tools and design patterns.
  \item The \textit{flexibility} of our process stems from the policy-based design and
    hierarchical structure ranging from low level to higher level interfaces.
    A user of our library can easily change any component of our framework.
  \item \textit{Reliability} was addressed by adding a carefully crafted autamated
    tests suite (see Section~\ref{section-testing}).
  \item We made sure our framework is \textit{relevant} to MCMC researchers
    by complementing it with MCMC
    output analysis tools described in Chapter~\ref{output-analysis-framework} and
    Appendix~\ref{mcmc-output-analysis}.
    We further showed in Chapter 5 how our framework can help to solve some of
    the current research problems (Figure~\ref{zig-zag-vs-bps} and Figure~\ref{refresh-rate-how-to-fit}).
  \item Finally, in Chapter~\ref{chapter-mcmc-applications} we demonstrated that
    the current PDMP based Monte Carlo algorithms are \textit{very easy to implement}
    by using our library.
\end{enumerate}

\noindent
We are by no means trying to suggest that our framework is perfect.
There is indeed a lot of more work to be done.
For example, an abstract layer of standard benchmarking problems could be
added (i.e., simulating from Gaussian distributions or solving a Bayesian
logistic regression problem).
Newly developed algorithms would then imediatelly get access to these
problems.
As a result, no additional code at all would
have to be written to compare it against the other implemented algorithms
(like the BPS or the Zig-Zag process).

Recall that one of the best features in our output analysis framework
is the ability to implement output processors which do the calculations online.
While due to memory limitations discussed in Section~\ref{section-output-processors}
it is necessary to do analysis online for extremely long runs, it is, however,
not necessary for short simulations.
In such a scenario analysing the output using a language like C++ is clearly
undesirable.
This problem could be completely solved by implementing an R interface to our
framework.

\section{Reflection on the Development Process}

At the begining of the project I was accustomed to solving every programming problem I had
by using inheritance.
Being new to C++, I did not initially realise how complex this language is and
how many different ways of doing the same things it offers.
The implementation of our framework, as outlined in Chapter~\ref{pdmps-implementation},
was already my second attempt.
My first purely object-oriented design resulted in too deep class hierarchies
and in the end reusing code was hard.
This is when I came accross the book by \citet{alexandrescu2001modern}
where the policy-based design was introduced, which seems to be the perfect
fit for our problem.

Writing a big test suite also played an important role in the development process.
It allowed us to catch many bugs, which would have otherwise been hard to track down.
Also, C++ compilers ignore template class methods that are never called.
To avoid surprises, it is important to have some code which calls all of our implemented
methods, to make sure that all of our framework compiles.

Finally, this project has served me as a great introduction to the Monte Carlo methods.
Designing the framework and thinking about code reuse helped me to better understand
the underlying theory.
I hence hope that this report can also serve as an accessible introduction to
PDMPs to a non-expert in the field.

It would be nice, however, if we did not have to worry about defining the dependencies
graph (from Chapter~\ref{pdmps-implementation}) nodes ourselves and only think in
terms of the factors that we add to the generator.
We thus implement a higher level \textit{builder} class, which takes a \textit{distribution}
as input and adds a relevant term to the generator of our process.
Also, we add Stan's mathematics library, which implements reverse-mode automatic differentiation
and many common probability distribution functions so that we need not to worry about these
details.


\end{document}
