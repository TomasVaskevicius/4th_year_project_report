\documentclass[report.tex]{subfiles}

\begin{document}

\chapter{Building a Framework for Output Analysis}
\label{output-analysis-framework}

In the previous chapter we designed and implemented a flexible and efficient
library for creating arbitrary PDMPs.
Assume that a user of our library wants to test a new MCMC algorithm based on
a PDMP.
A natural research workflow could then go as follows: first, we build the process
using our framework, then, we output the results to a text file and finally, we load
them with, for example, R or Python to analyse the output.
The shortcomings of such an approach are easy to see --
by outputting the simulation results to a text file we lose all the information
about the state space representation and what the deterministic flow function of the
process is.
This information is necessary in order to recreate the complete process trajectory (from the points at event times)
so it would have to be reprogrammed in the tool used for analysing the output.
Furthermore, due to the inherent randomness of our process, doing statistical
analysis requires simulating the same process a number of times.
These days, even the cheapest laptop computers have at least 4 logical processors, with the more
expensive ones reaching 8.
It would therefore be nice, if there was some easy way to use all the cores to run
independent processes, turning an experiment that could take a few days into a
simple overnight simulation.
The message is clear - our PDMPs framework on its own is not that useful.
In this chapter, we will extend our library with support for analysing the
output\footnote{The code for this chapter can be found at the project's GitHub
repository int the src/analysis/ directory.}.
Once again, our main focus is flexibility -- we want out output analysis utilities
to be easily extendable and customisable by the client for its special needs.

\section{Analysis utilities}
Let's first discuss the simplest tools that are necessary to do any kind of analysis
on the output of our processes.

\begin{enumerate}
  \item \textit{Support for running multiple processes in parallel}. This is very
    simple to do, since our separate processes will be independent from each other
    and will require no communication.
    Since the C++11 language standard, support for multithreading was added.
    We use it to write a simple function,
    which accepts a set of callable objects (e.g. lambda functions),
    executes them all in parallel\footnote{To avoid unnecessary overhead
    (e.g. context switching), we of course do not run more threads at a time
    than the number of available logical processors.} and collects the results.

  \item \textit{Support for visual analysis}.
    Since there is no built-in support for plotting graphs in C++, we use
    QCustomPlot \cite{qcustomplot-library} -- an open source library, built on top of Qt
    \cite{qt-library}, for plotting and data visualisation.
    We added wrappers around this library for easy plotting of line graphs,
    box plots and our simulated process' sample path (for the 2-dimensional case only).

  \item \textit{Support for numerical integration}.
    Recall, that the main motivation behind our framework is MCMC algorithms.
    As discussed in Chapter~\ref{mcmc-motivation}, we will want to use the output
    of our process to compute certain expectations.
    Since our algorithms output continuous trajectories rather than discrete samples
    from the target distribution, summation in the strong law of large numbers estimator
    becomes integration. We hence need tools for efficient one dimensional integraton.
    As a result, we added the GNU Scientific Library (GSL) \cite{gsl-library} --
    a C library for scientific computing
    which has an implementation of Gaussian quadrature.
    We implemented wrappers around the C API for easy and idiomatic C++ use.

  \item \textit{Support for timing}.
    Roughly speaking, one algorithm is better than the other, if it performs
    the same task faster. Hence it is natural that we need tools to easily time
    (possibly specific parts only) of our processes. Care must be taken when choosing the
    timing technique. One possibility is to use the wall-clock time.
    It is, however, very susceptible to noise due to operating system's activities.
    Another option is to use the CPU time. It has a downside, however, that an
    algorithm effectively using many cores and finishing much faster than another
    algorithm using one core only, may have, for instance, equal CPU times.
    All the algorithms in our project will run on one core only. We will want to run
    many independent processes at the same time using separate threads. Hence, we
    provide a wrapper around a timer measuring CPU time separately for each thread.
    Clients of our library may implement their own timers and use them with the existing code
    by simply changing template parameters where appropriate.
\end{enumerate}

\section{Running the process}

So far, our PDMPs' host class interface only supports simulation of one iteration
of the process.
Every user of our library thus has to write their own code for running the whole process
and deciding when to stop. This is clearly unacceptable, since many people would be
writing essentially the same code.
To solve this, we implement a separate class capable of running any PDMP\footnote{
  The runner class is general enough to run any kind of PDMP, not even necessarily built
  using our own framework. We could thus wrap any other PDMP implemented in C++ to return
  results compatible with our framework, and such third-party algorithm could fully benefit
  from our analysis framework's functionality.
}. We again reuse the policy-based design, by creating a host class for all runners
and delegating the specific running and stopping decisions to its policy class.
The host class will be responsible for registration of \textit{output processors}, which we will
introduce in the next section. We provide two running policies -- one for simply
performing a fixed number of iterations, another for stopping the process after
the specified amount of time (with extra support for the
\textit{burn-in}\footnote{Running a Markov chain
for a while and discarding its output. We hope that our process has converged to the invariant distribution
during this time.
} time and optionally excluding the time spent by the output processors.


\section{Output processors}
This is the most interesting decision we make in the output analysis framework's design.
To the best of our knowledge, most of MCMC libraries' APIs only support generating
a pre-specified number of samples from the target probability distribution.
Suppose we also did that and that we want to generated samples over $10^{4}$ dimensional space
using single precision floating point numbers.
Then, generating $25000$ samples would take at least $1$ GB of RAM. Running 8 such
chains simultaneously, would require $8$ GB of RAM, which is the limit in most
current laptop computers at the time. This puts severe limitations on the simulations that we can perform (without
outputting samples to a hard drive).

Note, however, that we do not actually need to save all the generated samples. Some of the most useful
output analysis statistics can be calculated by simply looping through the generated samples
only once. To solve our memory issue we can simply do our calculations while the process is running
and discard the old samples which allows our analysis framework to use only $O(1)$ memory.
We realise this idea by employing the \textit{Observer pattern} (see \citet{gamma1994design})
-- we create a base class called \textit{observer} for classes implementing arbitrary
calculations on the output of our process. Any class that inherits from the observer can register
to our PDMPs runner class.
Then, whenever the runner gets a new sample from the PDMP, it notifies all the observers
with the obtained new value, thus allowing them to do the calculations online.
Note that this design does not put any restrictions on the output analysis as one can implement
an observer, which simply stores all the output into an array, or writes the output into
a text file, which would allow us to use the standard analysis techniques.

We implement three output processors in our analysis framework -- expectation, autocorrelation
and asymptotic variance calculators. The last two are quite tricky to implement online
and require some compromise. We refer an interested reader to Appendix~\ref{mcmc-output-analysis}
for details on what autocorrelation and astymptotic variance is, why it
is useful in output analysis of MCMC algorithms and a brief explanation on how we calculate
these quantities online.
See Figure~\ref{figure-analysis-framework-design} for a graphical representation
of the analysis framework discussed in this chapter.

\begin{figure}
  \captionsetup{skip=1.25cm}
  \centering
  \def\svgwidth{\linewidth}
  \input{ch4-analysis-framework-uml.pdf_tex}
  \caption{A diagram representing the design of our output analysis framework.
           The output processors of choice have to register to the PDMP runner.
           The runner notifies all of the processors with latest iteration results of the PDMP which
           is being simulated.
           The running policy and output processors can use our library's analysis utilities (e.g. timers).}
  \label{figure-analysis-framework-design}
\end{figure}

\end{document}
